\documentclass[french, 12pt, letterpaper]{article}

\usepackage[utf8]{inputenc}

\usepackage[sfmath]{kpfonts} %% sfmath option only to make math in sans serif. Probablye only for use when base font is sans serif.
\renewcommand*\familydefault{\sfdefault} %% Only if the base font of the document is to be sans serif
\usepackage[T1]{fontenc}

\usepackage{lmodern}

\usepackage{xspace}
\usepackage{amsmath}
\usepackage{amsthm}
\usepackage{amssymb}
\usepackage{amsfonts}
\usepackage{hyperref}
\usepackage{amsthm}
\usepackage[french]{babel}
\usepackage{color}
\usepackage{listings}
\usepackage{bookmark}
\usepackage[subsection]{algorithm}
\usepackage{algorithmic}
\usepackage[nottoc]{tocbibind}

\newtheoremstyle{definition}
{15pt}% measure of space to leave above the theorem. E.g.: 3pt
{15pt}% measure of space to leave below the theorem. E.g.: 3pt
{}% name of font to use in the body of the theorem
{}% measure of space to indent
{\bfseries}% name of head font
{:}% punctuation between head and body
{\newline}% space after theorem head; " " = normal interword space
{}

\newtheoremstyle{proposition}
{15pt}% measure of space to leave above the theorem. E.g.: 3pt
{15pt}% measure of space to leave below the theorem. E.g.: 3pt
{}% name of font to use in the body of the theorem
{}% measure of space to indent
{\bfseries}% name of head font
{:}% punctuation between head and body
{\newline}% space after theorem head; " " = normal interword space
{}

\newtheoremstyle{example}
{15pt}% measure of space to leave above the theorem. E.g.: 3pt
{15pt}% measure of space to leave below the theorem. E.g.: 3pt
{}% name of font to use in the body of the theorem
{}% measure of space to indent
{\bfseries}% name of head font
{:}% punctuation between head and body
{\newline}% space after theorem head; " " = normal interword space
{}

\theoremstyle{definition}
\newtheorem{definition}{Définition}[subsection]

\theoremstyle{proposition}
\newtheorem{proposition}[definition]{Proposition}

\theoremstyle{example}
\newtheorem{example}{Exemple}[definition]

\bibliographystyle{abbrv}

\definecolor{bg_color}{rgb}{0.95, 0.95, 0.95}
\definecolor{comment_color}{rgb}{0,0.6,0}
\definecolor{string_color}{rgb}{0.58,0,0.82}
\definecolor{keyword_color}{rgb}{0.2, 0.1, 0.8}

\lstset{ %
  backgroundcolor=\color{bg_color},   % choose the background color
  basicstyle=\footnotesize,        % size of fonts used for the code
  breaklines=true,                 % automatic line breaking only at whitespace
  captionpos=b,                    % sets the caption-position to bottom
  commentstyle=\color{comment_color},    % comment style
  escapeinside={\%*}{*)},          % if you want to add LaTeX within your code
  keywordstyle=\color{keyword_color},       % keyword style
  stringstyle=\color{string_color},     % string literal style
}

\title{Recherche de chemins optimaux pour multiples agents}
\author{Ricou Jules}
\date{Mai 2020}

\begin{document}

    \maketitle

    \begin{abstract}
        Cette étude s'incrit dans le cadre de mon stage de fin de Licence à l'Université d'Angers. 
        Son but est la résolution du problème de recherche de chemins pour plusieurs agents
        sans qu'ils n'occupent une même position.
        L'encodage de ce sujet en un problème de satisfiabilité a pour objectif 
        de minimiser le nombre d'étapes maximales pour que tous les agents aient atteint leur destination.
    \end{abstract}

    \newpage

    \tableofcontents

    \newpage
    \section*{Introduction}
    \addcontentsline{toc}{section}{Introduction}

    La recherche de chemins pour multiples agents ou recherche coopérative de chemins consiste 
    à déplacer plusieurs agents sans qu'ils n'entrent en collision.
    Chaque agent démarre à une position et doit en atteindre une autre. 
    Plus précisement, chacun doit simultanément se déplacer sans être au même moment au même endroit.
    Ce problème vise à trouver une solution optimale.
    Sachant que le terme optimal aura, dans cette recherche, pour signification de minimiser la distance du chemin
    le plus long (\ref{sec:makespan}).

    La résolution de ce problème est NP-difficile. 
    La vérification d'une solution prend un temps polynomial mais trouver une solution se heurte à un nombre exponentiel
    de cas suivant le nombre d'agents et la taille de l'environement.

    Dans ce cas, il est sensé de convertir le problème originel en un problème de satisfiabilité
    pour bénéficier de la rapidité des solveurs SAT modernes. 

    Ce travail a pour vocation de modéliser ce problème grâce un encodage simple mais efficace \cite{PSUR16} (\ref{sec:encoding})
    et d'optimiser l'espace de recherche à l'aide d'un arbre décisionnel (\ref{sec:mdd}).
    Puis d'implémenter cet encodage et de le faire résoudre par un solveur SAT.

    \newpage
    \section{Contexte}

    Pour modéliser ce problème, l'environnement dans lequel evoluent les agents sera représenté par un graphe non orienté. 
    Chaque agent est considéré comme une entité discrètes positionnée sur un unique noeud du graphe.
    Le temps sera modélisé par une séquence d'intervalle régulier; à chaque interval, 
    tout agent pourra se déplacer -- ou non -- vers un noeud voisin en suivant une arête du graphe.
    La contrainte de conflit impose qu'à aucun moment, deux agents ne se retrouvent sur un même noeud.
    Le but de ce problème est de trouver, pour chaque agent, un chemin de son noeud initial à son noeud final, en un temps minimal. 

    \subsection{Modélisation formelle du problème}

    Soit $G = ( V, E )$, un graphe non orienté modélisant l'environement avec $V = \{v_0, v_1, \dots, v_n\}$ un ensemble fini de noeuds et 
    $E \subseteq \{\{v_i, v_j\}|v_i, v_j \in V\}$ un ensemble d'arêtes entre deux noeuds du graphe. 
    Ainsi que $A = \{a_1, a_2, \ldots, a_m\}$ un ensemble fini d'agents, $m \leq n$. 
    Le temps, comme décrit plus tôt, est divisé en intervals de temps discret $t$.

    On note $\alpha_t : A \rightarrow V$ un arrangement des agents à l'intervalle de temps $t$, 
    de telle sorte que $\alpha_t(a)\in V$ désigne le noeud qu'occupe l'agent $a \in A$. 
    Un seul agent peut être positionné sur un noeud.
    Par conséquent, $\alpha_t$ est uniquement inversible, et défini par $\alpha^{-1}_t(v) \in A \cup \{\bot\}$.
    Intuitivement, $\alpha^{-1}_t$ désigne l'agent occupant le noeud $v \in V$ à l'instant $t$, ou $\bot$ si le noeud est vide.

    \begin{definition}[Recherche coopérative de chemin]
        Une instance d'un problème de recherche coopérative de chemin, ou CPF \footnote{CPF: Cooperative pathfinding}, sera déclarée comme le quadruplé $\Sigma = ( G, A, \alpha_0, \alpha^+ )$;
        avec $\alpha_0$ et $\alpha^+$, la configuration de départ et, respectivement, de fin des agents.
        Une solution à cette instance est une séquence d'arrangements $[\alpha_0, \alpha_1, \ldots, \alpha_\tau]$ tel que $\alpha_\tau = \alpha^+$ 
        et $\alpha_{t+1}$ est une transition valide depuis $\alpha_t$ pour $t=0, 1, \ldots, \tau - 1$.
        
        Une transition entre deux arrangements $\alpha_i$ et $\alpha_j$ est valide, si et seulement si, elle vérifie ces deux propriétés:

        \begin{itemize}
            \item[] 
                \begin{gather}
                    \label{def:deplacement}
                    \forall a \in A, \alpha_i(a) = \alpha_j(a) \lor \{\alpha_i(a),\alpha_j(a)\} \in E
                \end{gather}
                \textbf{(Un agent ne se déplace pas ou suit une arête du graphe)}
            \item[] 
                \begin{gather}
                    \label{def:croisement}
                    \forall a, b \in A, a \neq b \implies \{\alpha_i(a), \alpha_j(a)\} \neq \{\alpha_i(b), \alpha_j(b)\}
                \end{gather}
                \textbf{(Deux agents ne se croisent pas sur une même arête)}
        \end{itemize}
    \end{definition}

    \begin{definition}[Durée d'une solution]
        \label{sec:makespan}
        La durée d'une solution est le temps nécéssaire pour que tous les agents aient atteint leur destination. 
        Pour une solution $[\alpha_0, \alpha_1, \ldots, \alpha_\tau]$, sa durée est égale à $\tau$.
        Il est aussi possible de la calculer par $max_{a \in A}\; \xi(a)$ où $\xi(a) \in \mathbb{N}$ est le nombre d'arête traversé par l'agent $a$.
    \end{definition}

    \subsection{Problème de satisfiabilité}

    Un problème SAT ou problème de satisfiabilité, est un problème de décision qui determine s'il existe une assignation de variable
    rendant une formule de logique propositionelle vraie.

    \begin{definition}[Formule de logique propositionelle]
        Soit l'ensemble de valeurs booléennes $\mathbb{B} = \{\bot, \top\}$, et $\mathbf{B}$ un ensemble infini et dénombrable de variables à valeurs dans $\mathbb{B}$.
        L'ensemble $\mathbf{P}$ des formules de logique propositionelle est construit comme suit:
        \begin{itemize}
            \item Si $b \in \mathbb{B}$ alors $b \in \mathbf{P}$
            \item Si $b \in \mathbf{B}$ alors $b \in \mathbf{P}$
            \item Si $p \in \mathbf{P}$ alors $\lnot p \in \mathbf{P}$
            \item Si $P, Q \in \mathbf{P}$ alors $P \land Q \in \mathbf{P}$
            \item Si $P, Q \in \mathbf{P}$ alors $P \lor Q \in \mathbf{P}$
            \item Si $P, Q \in \mathbf{P}$ alors $P \implies Q \in \mathbf{P}$
        \end{itemize}
    \end{definition}

    \begin{definition}[Assignation]
        Pour une formule $P \in \mathbf{P}$ contenant les variables booléennes $B \subset \mathbf{B}$.
        Une assignation $\Gamma : \mathbf{B} \rightarrow \mathbb{B}$ attribue une valeur booléenne de $\mathbb{B}$ à chaque variable de la formule.
        On note $\Gamma(P) \in \mathbf{P}$ la formule qui substitue à chaque variable $b \in B$ de $P$ à sa valeur booléenne $\Gamma(b)$.
    \end{definition}
    \begin{example}
        Soit $a, b, c \in \mathbf{B}$, la formule $P = a \land (b \lor \lnot c)$, et l'assignation $\Gamma = \{(a \rightarrow \bot), (b \rightarrow \top), (c \rightarrow \bot)\}$.
        Alors $\Gamma(P) = \bot \land (\top \lor \lnot \bot) = \bot$.
    \end{example}

    \begin{definition}[Satisfiabilité d'une formule]
        Une formule $P \in \mathbf{P}$ est satisfiable s'il existe une assignation $\Gamma$ tel que $\Gamma(P) = \top$.
        On notera alors $\Gamma \vDash P$.
    \end{definition}

    \newpage
    \section{Traduction en problème de satisfiabilité}

    Obtenir une solution optimale (du point de vue de la durée de la solution) est un problème NP, il peut donc être traduit en une instance
    d'un problème SAT comme le stipule le théorème de Cook. La formule propositionelle $F(\Sigma, \tau)$ est satisfiable si et seulement s'il existe une solution à $\Sigma$
    de durée maximale $\tau$.

    \subsection{Méthode de recherche de la solution optimale}
    \label{sec:methiter}

    Il existe plusieurs méthodes pour obtenir la solution optimale, la plus simple, et pourtant efficace, consiste à séquentiellement construire
    et résoudre la formule $F(\Sigma, \tau)$ pour $\tau = 0, 1, \ldots$ jusqu'à ce que $F(\Sigma, \tau)$ soit satisfiable 
    (Pseudo-code de la stratégie est décrit dans \ref{alg:par}).
    Il est à noter que cette méthode ne termine jamais s'il n'existe pas de solution à $\Sigma$. Pour éviter ce cas, il est possible 
    d'utiliser un algorithme polynomial tel que PUSH-AND-SPIN \cite{PAS} (\ref{sec:par}) vérifiant l'existence d'une solution.

    \begin{algorithm}
        \caption{Trouve sequentiellement la solution avec la plus petite durée qui résout $\Sigma$. 
        Si aucune solution n'est possible, $\emptyset$ est retourné.}
        \begin{algorithmic}
            \label{alg:par}
            \IF{$\Sigma \text{ a une solution}$}
                \STATE $\tau \leftarrow 0$
                \LOOP
                    \STATE $F(\Sigma, \tau) \leftarrow \text{encode vers SAT}(\Sigma, \tau)$
                    \IF{$\text{resoud}(F(\Sigma, \tau))$}
                        \STATE $s \leftarrow \text{extrait solution}(F(\Sigma, \tau))$
                        \RETURN $( s, \tau )$ 
                    \ENDIF
                \ENDLOOP
            \ELSE
                \RETURN $( \emptyset, \infty )$ 
            \ENDIF
        \end{algorithmic}
    \end{algorithm}
        
    \subsection{Expansion d'un graphe dans le temps}

    Un chemin dans le temps d'un agent sur le graphe $G$ n'est pas nécessairement simple, il peut utiliser plusieurs fois un même noeud.
    Pour résoudre ce problème, le graphe $G$ sera étendu dans le temps. 
    Chaque noeud aura $\tau$ synonymes représentant la visite du noeud à l'instant $t$.
    Ainsi à chaque agent pourra être attribué un chemin simple dans le graphe étendu.

    \begin{definition}[Expansion d'un graphe dans le temps $TEG^\tau(G)$]
        Soit $G = ( V, E )$ un graphe non-orienté et $\tau\in \mathbb{N}$,
        une expansion dans le temps du graphe $G$ avec $\tau + 1$ couche de temps (indéxé de $0$ à $\tau$) est un graphe orienté
        $TEG^\tau(G) = ( V', E' )$; avec:
        \begin{align*}
            V' = &\{v^t\;|\;v \in V \land t = 0, 1, \ldots, \tau\}\\
            E' = &\{( u^t, v^{t+1} ), ( v^t, u^{t+1} )\;|\;\{u, v\}\in E \land t= 0, 1, \ldots, \tau - 1\} \cup\\
            &\{( v^t, v^{t+1} )\;|\;v\in V \land t= 0, 1, \ldots, \tau - 1\}.
        \end{align*} 
    \end{definition}

    La recherche coopérative de chemin de durée $\tau$ peut être considéré comme une recherche d'un ensemble de chemins disjoints et sans croisement
    dans le graphe étendu dans le temps.

    \begin{definition}[Ensemble de chemins disjoints et sans croisement]
        Soit l'ensemble de chemins $\Pi = \{\pi_0, \pi_1, \ldots, \pi_m\}$ dans $TEG^\tau(G) = (V', E')$,
        tel que pour tout $i \leq m$, $\pi_i$ connecte $v^0 \in V'$ à $u^\tau \in V'$, $v, u \in V$,
        c'est-à-dire, $\pi_i$ est une séquence jointe d'arêtes dans $E'$. 
        Cet ensemble $\Pi$ est dit disjoint et sans croisement si et seulement si
        pour toutes paires de chemins $\pi_i$ $\pi_j$, $i \neq j$:
        \begin{gather}
            \label{def:chemindisjoint}
            \begin{split}
            &V(\pi_i) \cap V(\pi_j) = \emptyset\\
            \text{avec } &V(\pi_k) \subseteq V',\text{ l'ensemble des noeuds parcourus de }\pi_k\\
            &\textbf{(Les noeuds parcourus annotés du temps)}
            \end{split}
        \end{gather}
        \begin{gather}
            \label{def:chemincroisement}
            \begin{split}
            &TE(\pi_i) \cap TE(\pi_j) = \emptyset\\
            \text{avec } &TE(\pi_k) = \{\{v_i, v_j\}^t_{t+1}  \;|\; ( v_i^t, v_j^{t+1} ) \in \pi_k \}\\
            &\textbf{(Les arêtes (non orientées) parcourus annotées du temps)}
            \end{split}
        \end{gather}

        La propriété (\ref{def:chemindisjoint}) assure que deux chemins de $\Pi$ n'occupent un même noeud. 
        Et (\ref{def:chemincroisement}) garanti qu'aucun croisement n'est possible.
    \end{definition}

    \begin{proposition}
        Une solution à un problème CPF $\Sigma = ( G, A, \alpha_0, \alpha^+ )$ avec $A = \{a_1, a_2, \ldots, a_m\}$ et de durée $\tau$,
        existe si et seulement s'il existe un ensemble $\Pi = \{\pi_0, \pi_1, \ldots, \pi_m\}$ de chemins disjoints et sans croisement dans $TEG^\tau(G)$
        tel que $\pi_i$ connecte $\alpha_0(a_i)^0$ à $\alpha^+(a_i)^\tau$ pour $i = 1, 2, \ldots, m$.
    \end{proposition}

    \begin{proof}
        Soit un problème CPF $\Sigma = ( G, A, \alpha_0, \alpha^+ )$ et une de ses solutions 
        $\mathcal{S} = [\alpha_0, \alpha_1, \ldots, \alpha_\tau]$, avec
        $A = \{a_1, a_2, \ldots, a_m\}$ et $G = ( V, E )$.
        
        L'ensemble des chemins $\Pi = \{\pi_0, \pi_1, \ldots, \pi_m\}$ de $TEG^\tau(G)$ disjoints et 
        sans croisement peut être construit à partir de $\mathcal{S}$ en définissant le chemin
        $\pi_i$ comme la trajectoire de l'agent $a_i \in A$. Plus particulièrement: 
        \begin{gather*}
            \pi_i = [( \alpha_0(a_i)^0, \alpha_1(a_i)^1 ), ( \alpha_1(a_i)^1, \alpha_2(a_i)^2 ),
            \ldots, ( \alpha_{\tau-1}(a_i)^{\tau-1}, \alpha_\tau(a_i)^\tau )]
        \end{gather*}
        De plus, $\pi_i$ est un chemin de $TEG^\tau(G)$ car, étant donné que $\{\alpha_t(a_i), \alpha_{t+1}(a_i)\} \in E$, 
        $( \alpha_t(a_i)^t, \alpha_{t+1}(a_i)^{t+1} ) \in E'$ 
        par construction de $TEG^\tau(G) = ( V', E' )$. 
        Evidemment, $\pi_i$ connecte $\alpha_0(a_i)^0$ à $\alpha_\tau(a_i)^\tau$.
        Il ne reste plus qu'à vérifier que les chemins soient disjoints et sans croisement.
        On sait que $V(\pi_i)$ est l'ensemble $\{\alpha_0(a_i)^0, \alpha_1(a_i)^1, \ldots, \alpha_\tau(a_i)^\tau\}$, or, un arrangement ne peut affecter deux agents
        sur un même noeud, par conséquent (\ref{def:chemindisjoint}) est respecté.
        Aucun croisement ne peut avoir lieu (\ref{def:chemincroisement}) sinon la transition entre deux arrangements (\ref{def:croisement}) ne serait pas respectée.

        Montrons maintenant qu'il est possible de construire une solution du problème $\Sigma = ( G = ( V, E ), A, \alpha_0, \alpha^+ )$ de durée $\tau$ 
        à partir d'un ensemble de chemins $\Pi = \{\pi_0, \pi_1, \ldots, \pi_m\}$ de $TEG^\tau(G)$ disjoints et sans croisement 
        où chaque chemin $\pi_i$ connecte $\alpha_0(a_i)$ à $\alpha_+(a_i)$. 

        Assumons l'existence d'un tel ensemble. 
        On défini alors $\alpha_t(a_i) = v_l$ avec $\pi_i = [( v_0^0, v_1^1 ), ( v_1^1, v_2^2 ), \ldots, 
        ( v_{p-1}^{\tau-1}, v_p^\tau )]$, $v_l \in V$, $l = 0, 1, \ldots, p$, $t = 0, 1, \ldots, \tau$.
        Les chemins sont disjoints, donc les arrangements sont bien formés. 
        De plus, par construction de $TEG^\tau(G) = ( V', E' )$, pour toute arête orientée $( v^t, u^{t+1} ) \in \pi_i$ 
        implique que $v = u$ ou $\{v, u\} \in E$ (\ref{def:deplacement}). 
        Enfin aucun chemin de $\Pi$ ne se croise (\ref{def:chemincroisement}), impliquant nécéssairement que toutes transitions entre deux arrangements
        soient valides (\ref{def:croisement}).


    \end{proof}

    \subsection{Réduire l'espace de recherche}
    \label{sec:mdd}

    Cette section emprunte l'idée d'un MDD \footnote{MDD: Multi-value decision diagram} de l'algorithme ICTS \cite{ICTS} pour réduire l'espace de recherche. 
    Cela permet d'améliorer considérablement la vitesse de résolution du problème (\ref{sec:tailleclause}) et
    de réduire le nombre de variables et la taille des clauses.
    Intuitivement, le MDD d'un agent est le graphe n'incluant que les noeuds accessibles dans le temps imparti.

    \begin{definition}[Reduction de l'espace de recherche d'un agent $MDD_i^\tau$]
        Soit le problème CPF $\Sigma = ( G, A, \alpha_0, \alpha^+ \rangle$ avec $G = \langle V, E )$, $A = \{a_1, a_2, \ldots a_m\}$.
        Lors de la recherche d'une solution de durée $\tau$, on notera $MDD_i^\tau \subseteq TEG^\tau = ( V', E' )$ l'environement de l'agent 
        $a_i \in A$, $i = 1, 2, \dots, m$. Plus précisement, sachant que $\xi(v, u)$ est le coût de déplacement minimal entre deux noeuds $v, u \in V$, 
        $MDD_i^\tau = ( V^-, E^- )$ avec:
        \begin{align*}
            V^- =& \{v^t\;|\; \xi(\alpha_0(a_i), v) + \xi(v, \alpha^+(a_i)) \leq \tau, v^t \in V'\}\\
            E^- =& \{( v, u ) \;|\; ( v, u ) \in E',\ v, u \in V^-\}
        \end{align*} 
    \end{definition}

    \begin{proposition}
        S'il existe une solution de durée $\tau$ pour un problème CPF $\Sigma = ( G, A, \alpha_0, \alpha^+ )$,
        avec $G = ( V, E )$
        et donc un ensemble de chemins dans $TEG^\tau(G)$ disjoints et sans croisement $\Pi = \{\pi_0, \pi_1, \ldots, \pi_m\}$ alors tout chemin
        $\pi_i \in \Pi$ est dans $MDD^\tau_i$.
    \end{proposition}

    \begin{proof}
        Supposons l'existence d'un chemin $\pi_i$ de l'ensemble $\Pi$ qui n'est pas contenu dans $MDD_i^\tau$.
        Alors il existe un $v \in V(\pi_i)$ tel que $\xi(\alpha_0(a_i), v) + \xi(v, \alpha^+(a_i)) > \tau$, $a_i \in A$.
        Or $\pi_i$ est un chemin dans $TEG^\tau(G)$, c'est-à-dire de longueur $\tau$.
        Donc tous chemins $\pi_i \in \Pi$ est un chemin dans $MDD_i^\tau$. 
    \end{proof}

    

    \subsection{Encodage en problème de satisfiabilité}
    \label{sec:encoding}

    Pour transformer le problème de recherche de chemins, le concept de graphe étendu dans le temps est très important car
    il permet de représenter tous les arrangements possibles des agents à tout temps.
    L'encodage utilisé s'inspire de l'encodage DIRECT \cite{PSURDIRECT}, à l'exception que nos prémisses diffèrent légèrement:
    notre modèle autorise qu'un agent entre dans un noeud occupé si celui-ci contient un agent qui en sort dans le même interval de temps.

    \begin{definition}[Encodage DIRECT $F(\Sigma, \tau)$]
        \label{def:encoding}
        Soit $\Sigma=( G, A, \alpha_0, \alpha_+)$ une instance d'un problème CPF avec $G = ( V, E )$ et $\tau\in \mathbb{N}$
        la durée maximale, l'expansion de ce graphe, et sa réduction pour chaque agent $a_i \in A$ nous donne $MDD^\tau_i(G) = ( V', E' )$.
        Pour encoder ce problème, on défini les variables $\chi_{a, v}^t$ comme représentant la présence de l'agent $a\in A$ sur le noeud $v^t \in V'$.
        Les contraintes suivantes assurent la validité du modèle défini:
        
        \begin{itemize}
            \item[] 
                \begin{gather}
                    \begin{split}
                        \chi_{a, v}^t \implies \bigvee_{u\;|\;( v^t, u^{t+1} ) \in E'} \chi_{a, u}^{t+1}
                        \\\forall a \in A, \forall v \in V, \forall t \in \{0, 1, \ldots, \tau-1\}
                    \end{split}
                \end{gather}
                \textbf{(Un agent suit une arête du graphe)}
            \item[] 
                \begin{gather}
                    \begin{split}
                        \chi_{a, v}^t \implies \bigwedge_{u \in V, u \neq v} \lnot \chi_{a, u}^t
                        \\\forall a \in A, \forall v \in V, \forall t \in \{0, 1, \ldots, \tau\}
                    \end{split}
                \end{gather}
                \textbf{(Un agent est positionné sur un noeud au maximum par intervalle de temps)}
            \item[] 
                \begin{gather}
                    \begin{split}
                        \chi_{a, v}^t \implies \bigwedge_{b \in A, a \neq b} \lnot \chi_{b, v}^t
                        \\\forall a \in A, \forall v \in V, \forall t \in \{0, 1, \ldots, \tau\}
                    \end{split}
                \end{gather}
                \textbf{(Un noeud contient un agent au maximum par intervalle de temps)}
            \item[] 
                \begin{gather}
                    \begin{split}
                        \chi_{a, v}^t \land \chi_{a, u}^{t+1} \implies \bigwedge_{b \in A, a \neq b} \lnot (\chi_{b, u}^t \land \chi_{b, v}^{t+1})
                        \\\forall a \in A, \forall \{v, u\} \in E, \forall t \in \{0, 1, \ldots, \tau-1\}
                    \end{split}
                \end{gather}
                \textbf{(Deux agents ne se croisent pas)}
        \end{itemize}
    \end{definition}

    \subsection{Conversion en forme normale conjonctive}
    \label{sec:cnf}

    Afin d'utiliser un solveur SAT, les contraintes doivent être en forme normale conjonctive.

    \begin{definition}[Forme normale conjonctive]
        Une formule propositionelle est en forme normale conjonctive lorsque celle-ci est une conjonction de clauses; où
        chaque clause est une disjonction de littéraux.
        Un littéral est une variable ou une valeur booléenne, de plus la négation d'un littéral est également un littéral.
    \end{definition}

    \begin{example}
        Voici une liste non exhaustive de formules en forme normale conjonctive: $A \land (B \lor \lnot C)$, $A$, $(A \lor \lnot B) \land (\lnot A \lor C)$...; 
        ou d'autres qui ne respectent pas cette propriété $\lnot (A \lor B)$, $A \lor (B \land \lnot C)$... 
    \end{example}

    Transformer les contraintes précedentes (\ref{def:encoding}) en forme normale conjonctive est trivial grâce aux lois de De Morgan et de distributivité.
    Une implication $A \implies B$ pourra aussi être éliminée et remplacée par $\lnot A \lor B$.

    \begin{itemize}
        \item[] 
            \begin{align}
                \label{equ:contrainteA}
                \begin{split}
                    &\chi_{a, v}^t \implies \bigvee_{u\;|\;( v^t, u^{t+1} ) \in E'} \chi_{a, u}^{t+1}
                    \\\iff \qquad &\lnot \chi_{a, v}^t \lor \bigvee_{u\;|\;( v^t, u^{t+1} ) \in E'} \chi_{a, u}^{t+1}
                    \\&\forall a \in A, \forall v \in V, \forall t \in \{0, 1, \ldots, \tau-1\}
                \end{split}
            \end{align}
            \textbf{(Un agent suit une arête du graphe)}
        \item[] 
            \begin{align}
                \label{equ:contrainteB}
                \begin{split}
                    &\chi_{a, v}^t \implies \bigwedge_{u \in V, u \neq v} \lnot \chi_{a, u}^t
                    \\\iff \qquad &\bigwedge_{u \in V, u \neq v} (\lnot \chi_{a, v}^t \lor \lnot \chi_{a, u}^t)
                    \\&\forall a \in A, \forall v \in V, \forall t \in \{0, 1, \ldots, \tau\}
                \end{split}
            \end{align}
            \textbf{(Un agent est positionné sur un noeud au maximum par intervalle de temps)}
        \item[] 
            \begin{align}
                \label{equ:contrainteC}
                \begin{split}
                    &\chi_{a, v}^t \implies \bigwedge_{b \in A, a \neq b} \lnot \chi_{b, v}^t
                    \\\iff \qquad &\bigwedge_{b \in A, a \neq b} (\lnot \chi_{a, v}^t \lor \lnot \chi_{b, v}^t)
                    \\&\forall a \in A, \forall v \in V, \forall t \in \{0, 1, \ldots, \tau\}
                \end{split}
            \end{align}
            \textbf{(Un noeud contient un agent au maximum par intervalle de temps)}
        \item[] 
            \begin{align}
                \label{equ:contrainteD}
                \begin{split}
                    &\chi_{a, v}^t \land \chi_{a, u}^{t+1} \implies \bigwedge_{b \in A, a \neq b} \lnot (\chi_{b, u}^t \land \chi_{b, v}^{t+1})
                    \\\iff \qquad &\bigwedge_{b \in A, a \neq b} \lnot \chi_{a, v}^t \lor \lnot \chi_{a, u}^{t+1} \lor \lnot \chi_{b, u}^t \lor \lnot \chi_{b, v}^{t+1}
                    \\&\forall a \in A, \forall \{v, u\} \in E, \forall t \in \{0, 1, \ldots, \tau-1\}
                \end{split}
            \end{align}
            \textbf{(Deux agents ne se croisent pas)}
    \end{itemize}

    \subsection{Taille des clauses et nombre de variables}
    \label{sec:tailleclause}

    La modélisation du problème CPF a pour but d'être résolu par un solveur SAT, ainsi pour accélérer ce processus, prendre en compte le fonctionement
    des solveurs SAT est primordial. L'un des facteurs les plus importants est la taille de la formule, le nombre de clauses et de variables doit être minimal.
    Même si ce travail n'a pas pour but de comparer d'autres modélisations existantes, il reste important de calculer la taille de la formule générée.

    Pour une formule $F(\Sigma, \tau)$ d'une instance $\Sigma=( G, A, \alpha_0, \alpha_+)$ sur le graphe $G = ( V, E )$, 
    le nombre de variables est simple à calculer: $O(\tau \cdot |A| \cdot |V|)$. 
    Ensuite chaque contraintes ajoutent de clauses.
    La propriét de déplacement \ref{equ:contrainteA} contient $O(\tau \cdot |A| \cdot |V|)$ clauses.
    La propriété sur l'unicité d'une position par agent et par intervalle \ref{equ:contrainteB} ajoute $O(\tau \cdot |A|^2 \cdot |V|)$ clauses.
    La propriété sur l'unicité d'un agent par noeud par intervalle \ref{equ:contrainteC} ajoute $O(\tau \cdot |A| \cdot |V|^2)$ clauses.
    Et enfin, la contrainte qui impose l'absence de croisement (\ref{equ:contrainteD}) ajoute $O(\tau \cdot |A|^2 \cdot |V|)$ clauses.

    La réduction de l'espace de recherche entraîne une diminution de la taille de la formule.
    La topologie du graphe influence son impacte. 
    C'est pour cette raison qu'une comparaison sera faites sur des exemples concrets (\ref{sec:compartaille}). 

    \newpage
    \section{Implémentation}

    Maintenant que le problème a été modélisé, la mise en pratique consiste à developper un programme qui traduit un problème CPF en 
    un problème SAT et de le résoudre.
    La première étape est résolue par l'introduction d'un nouveau format de fichier (\ref{sec:filecpf}).
    La résolution du problème SAT est effectuée par un programme externe \emph{Glucose} (\ref{sec:satsolver}). 
    Et la traduction en problème SAT sera faite simplement en suivant les formes normales conjonctives précédement définies (\ref{sec:cnf}). 

    \subsection{Format de fichier CPF}
    \label{sec:filecpf}

    Afin de définir le graphe, l'arrangement initial et l'arangement final, un nouveau format de ficher est défini.
    Celui-ci est simple car le graphe n'est pas directionnel et les arêtes n'ont aucun poids.
    Pour plus de flexibilité, toute ligne commençant par "\#" est ignorée, servant ainsi de commentaire. 
    Autrement, le format est constititué d'une série de nombres séparés pas un espace. 
    En voici un exemple:
    
    \lstinputlisting[language=sh]{test.cpf}
        
    \subsection{Détermination de l'existence d'une solution}
    \label{sec:par}

    Pour déterminer si un problème CPF a une solution, la méthode de recherche itérative (\ref{sec:methiter}) ne fonctionne pas;
    dans le cas où aucune solution n'existe, cette méthode ne termine jamais.
    Pour compenser ce défaut, un algorithme supplémentaire tel que PUSH-AND-SPIN \cite{PAS} peut-être ajouté en début de programme.
    Cependant, son implémentation s'est montré trop complexe pour être dans la version finale du programme. 

    \subsection{Utilisation du solveur SAT \emph{Glucose}}
    \label{sec:satsolver}

    Pour résoudre le problème SAT, l'utilisation d'un solveur tel que \emph{Glucose} est primordial au vu de le compléxité de ce programme.
    Tel quel, \emph{Glucose} utilise un fichier d'entrée au format \emph{dimacs}.
    Afin d'éviter cet intermédiaire, \emph{Glucose} a été 
    légèrement modifié afin de pouvoir injecter directement les clauses.
    Cela permet aussi d'éviter une intilisation de \emph{Glucose} à chaque test de la méthode itérative (\ref{sec:methiter}).

    \subsection{Traduction du problème CPF en problème SAT}

    La création de la formule est triviale, une variable est créée pour chaque noeud du $TEG$ et les clauses sont ajoutées en
    suivant leurs définitions (\ref{sec:cnf}).

    La réduction de l'espace de recherche complexifie quelque peu le programme. 
    Pour déterminer les variables nécessairement fausses, deux parcours en largeur sont utilisés à partir
    du noeud initial et final de chaque agent.
    En effet, le parcours en profondeur permet non seulement d'être calculé itérativement mais aussi d'eviter le parcours du graphe dans sa totalité.

    Pour déterminer si une variable $\chi^t_{a, v}$ existe pour une durée maximale $\tau$, alors $t \ge \xi(\alpha_0(a), v)$ et
    $\tau - t \ge \xi(v, \alpha_+(a))$, $\xi(v, u)$ étant égale à la plus petite distance entre $v$ et $u$; ou, dans notre cas,
    la profondeur de $u$  par rapport à $v$. 
    Lors de la création d'une clause contenant une variable fausse, celle-ci est omise si elle contient la négation de cette variable.
    Autrement le terme contenant la variable est supprimé de la clause.


    \subsection{Utilisation du programme}

    Le solveur contient multiples arguments, tous décrits avec l'option \lstinline{--help} comme présenté ci-dessous.
    Le programme utilise le format CPF (\ref{sec:filecpf}) pour définir le problème CPF à résoudre. 
    Une option importante est \lstinline{--no-mdd} pour désactiver l'optimisation de réduction de l'espace de recherche dans un but de 
    comparaison. 
    
    \lstinputlisting[]{./usage/solver_help.txt}

    \subsection{Exemple d'exécution}
    \label{sec:example}

    Les deux exemples d'exécution dans les sous-sections suivantes ont été établis sur le problème défini dans le fichier
    \lstinline{impl/test/grid_test.cpf}\footnote{Voir annexe \nameref{anx:verifier}}. Celui-ci à été généré\footnote{Voir annexe \nameref{anx:generator}} 
    à partir d'une grille de taille 10 par 10;
    où chaque cellule a une probabilité de 10\% d'être un obstacle non traversable et 20\% d'être la cellule initiale d'un agent.
    À chaque agent a été assigné une destination aléatoire parmis les cellules sans obstacle.

    \subsubsection{Utilisation de la réduction de l'espace de recherche}

    Voici les résultats obtenus utilisant le programme avec les options: \lstinline{./solver --output=result_with_mdd.res_cpf --input=test/grid_test.cpf}.
    On peut remarquer que l'utilisation du MDD permet d'éliminer les itérations avec une durée non suffisante pour que tous les agents aient au moins un chemin.

    \lstinputlisting[]{./usage/solver_with_mdd.txt}

    \subsubsection{Désactivation de la réduction de l'espace de recherche}

    Voici les résultats obtenus utilisant le programme avec les options: \lstinline{./solver --output=result_without_mdd.res_cpf --input=test/grid_test.cpf --no-mdd}.

    \lstinputlisting[]{./usage/solver_without_mdd.txt}

    \subsubsection{Comparaison de la taille de la formule avec et sans la réduction de l'espace de recherche}
    \label{sec:compartaille}

    Le tableau suivant compare la résolution avec et sans la réduction de l'espace de recherche. Voici les exemples utilisés:
    \begin{itemize}
        \item \emph{Grand environement}: grille 12x12, ~15 obstacles et ~70 agents
        \item \emph{Beaucoup d'obstacles}: grille 6x6, ~4 obstacles et ~28 agents
        \item \emph{Beaucoup d'agents}: grille 6x6, ~15 obstacles et ~4 agents
    \end{itemize}
    
    \begin{center}
        \begin{tabular}{l|lll|lll|}
            \cline{2-7}
                & \multicolumn{3}{l|}{Avec MDD} & \multicolumn{3}{l|}{Sans MDD} \\ \cline{2-7} 
                & \multicolumn{1}{l|}{\#Var} & \multicolumn{1}{l|}{\#Clauses} & Temps & \multicolumn{1}{l|}{\#Var} & \multicolumn{1}{l|}{\#Clauses} & Temps \\ \hline
            \multicolumn{1}{|l|}{Grand environement} & 66k & 13.4M & 83s & 240k & 100M & 4100s \\ \hline
            \multicolumn{1}{|l|}{Beaucoup d'obstacles} & 1689 & 53177 & 382ms & 9072 & 11.7M & 1970ms \\ \hline
            \multicolumn{1}{|l|}{Beaucoup d'agents} & 53 & 208 & 1.4ms & 756 & 30k & 56ms \\ \hline
        \end{tabular}
    \end{center}
    
    L'utilisation d'un MDD améliore grandement l'éfficacité de l'algorithme en diminuant la taille de la formule par plusieurs
    ordres de grandeurs. Même si, pour certains problèmes, lorsque la durée maximale permet à tous les agents d'atteindre tous les noeuds
    du graphe, la taille de la formule est la même; le temps utilisé pour construire les MDDs est en partie compensé par les itérations précédentes
    prenant partie de cette optimisation. On peut remarquer que la vérification que tout agent possède au minimum un chemin pour 
    la durée demandée est un biproduit trivial lorsque le MDD est généré; éliminant totalement l'utilisation de solveur SAT pour cette itération. 

    \newpage
    \section*{Conclusion}
    \addcontentsline{toc}{section}{Conclusion}

    La formalisation d'un problème de recherche coopérative de chemins permet une compréhension détaillée du problème
    et une conversion en un problème de satisfiabilité sans souci. L'utilisation d'un graphe étendu dans
    le temps est essentielle pour générér une formule logique résolvant un problème CPF de durée maximale. 
    En augmentant la durée allouée, une solution optimale est obtenable.

    L'utilisation d'un solveur SAT moderne tel que \emph{Glucose} a permis d'accéder à de multiples optimisations bénéfique
    à la recherche d'une solution au problème. Le choix de l'encodage DIRECT a non seulement permis une modélisation simplifiée, comparé à d'autres
    méthodes \cite{PSUR16}, mais aussi une implémentation rapide et sans encombre. On a aussi remarqué que la réduction de l'espace
    de recherche par un MDD a contribué significativement à l'efficacité du programme.

    Finalement, peu de problèmes se sont imposés. Cependant, par manque de temps, l'algorithme PUSH-AND-SPIN \cite{PAS} n'a pas pu faire partie de
    l'implémentation finale. Même s'il n'est pas le point central du sujet, son ajout aurait sans aucun doute rendu ce programme
    plus simple d'utilisation.

    Pour conclure, ce travail s'est engagé à définir une méthode pour obtenir une solution optimale. 
    Seulement, minimiser la longueur du chemin le plus long a pour conséquence de ne pas contraindre la longueur des autres chemins
    de cette solution. Par exemple, une solution contenant deux chemins de longueur $n$ n'est pas moins optimale qu'une solution 
    avec un chemin de longueur $n$ et un autre vide. Pour résoudre ce problème, il est alors possible d'encoder une borne supérieure
    à la somme des longueurs des chemins \cite{PSURTEG} en plus de la borne supérieure sur la durée de la solution.

    Une optimisation supplémentaire peut-être aussi implémenté, en effet sur certaines instances d'un problème CPF, les agents peuvent
    être séparés en groupes indépendants, dans le but de réduire le nombre de clauses et leurs tailles.
    Pour sa réalisation, une division du graphe doit être présente afin qu'aucun agent d'une des deux parties ait pour destination
    un noeud de l'autre partie du graphe. De plus la division du graphe ne doit pas empêcher toutes les solutions optimales. 
    Sans un travail plus approfondi du sujet, cette dernière propriété est complexe à définir mais toutefois possible.
    
    \newpage
    \bibliography{biblio}

    \section*{Annexes}
    \addcontentsline{toc}{section}{Annexes}

    \subsection*{Utilisation du générateur de problème CPF}
    \label{anx:generator}

    Ce programme permet de générer un problème CPF au format du même nom.
    L'instance est constititué d'une grille de noeuds reliés a leurs voisins, sauf en cas d'obstacle.
    Les agents et les obstacles sont affecté alétoirement dans la grille.

    \lstinputlisting[]{./usage/generator_help.txt}

    \subsection*{Utilisation du vérificateur/afficheur de solution}
    \label{anx:verifier}

    Ce programme annexe permet de vérifier le bon fonctionement du programme principale.
    Si le graphe est une grille comme produit par le générateur (Voir annexe \nameref{anx:generator}) alors la solution est
    apte à être affiché dans la console.

    \lstinputlisting[]{./usage/verifier_help.txt}

\end{document}